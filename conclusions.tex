\chapter{Conclusions and Discussion}
\label{chap:Conclusions}
As demonstrated with a successful trial experiment with STAR, Kestrel
is capable of managing large-scale scientific applications in the
Cloud, with the condition that tasks are able to run independently
in a bag of tasks model. While Kestrel did not provide file transfer
capabilities itself, offloading the responsibility to existing infrastructure
such as HTTP servers proved to be a workable solution, handling over
seven terabytes of output data. While the VM images used were up to
twenty five gigabytes in size, KVM's snapshot mode reduced the startup
times for the VMs by only transferring image data when accessed. Snapshot
mode also prevented the main image from being modified by running
instances by writing changes to the VM host's local disk; the addition
of the task cleanup command to allow for restarting the VM prevented
exhausting the local hard drive's capacity when shared with other
VMs.

Current and future work with Kestrel will focus on using multiple
managers for a single pool. XMPP already provides the federation infrastructure
needed to operate a single pool with multiple XMPP servers, but more
work is needed to federate the manager components. Through the use
of a shared data store, such as Redis \cite{Redis}, a multi-manager
installation can be easily achieved so long as only a single XMPP
server is used since some implementations automatically load-balance
components \cite{ejabberd}. Linking managers on different servers,
in particular servers from different organizations, will require additions
to Kestrel's protocol to create the equivalent of {}``flocking''
between Condor systems. The current design under consideration is
to use a distributed hash table (DHT) to spread workers amongst managers,
and then allow managers to masquerade as a worker to other managers.
The masquerading manager would advertise the union of its worker's
capabilities, and would act as a task router, forwarding any received
tasks to one of its workers that match the task.
