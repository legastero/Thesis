\chapter{Introduction}
\label{chap:Introduction}

Cloud computing has sprung as a new paradigm for both enterprise and 
scientific application development and deployment. However, the term ``cloud
computing'' remains vague and broad in application, with considerations 
by Foster \cite{Foster2009} and Vaquero \cite{Vaquero2008} notwithstanding,
making it difficult to agree on a single, specific definition. Therefore, for
the purposes of this work, the term ``cloud computing'' will be deemed to
refer to the on-demand allocation of virtualized resources over a wide area
network. Within this model, a cloud is established through instantiating 
virtual machines from one or more virtualization enabled grid sites, or other
cloud providers, such as Amazon's EC2 service \cite{EC2}.

Such an infrastructure provides several benefits for scientific computing users.
Since the computational work will be done on a virtual machine instead of a
physical machine, users are able to package their own environment suitable for
the desired application by preparing a VM image based on a standard worker node
VM and customizing it with the all of the required supporting libraries and
programs needed for their application, independent of the available library
support of the physical host machines. By tailoring VMs to applications, it
is not as necessary for a job scheduler to perform complex matching between
jobs and the capabilities of worker agents, as performed through the ClassAd
match-making approach used by Condor \cite{Tannenbaum2001}; each job may simply
be tagged with the name of the VM configuration created for the specific
application. In addition VMs provide better application encapsulation compared
to regular operating system and cluster scheduling. Such encapsulation permits
improved resource utilization via consolidation mechanisms and improved security
by isolating user's applications from each other \cite{Figueiredo2003}. From
the resource owner's point of view, virtualized resources also enable greater
flexibility in the system administration because multiple operating systems can
be offered at the same time.

\section{Virtual Organization Clusters}
Such an arrangement for the use of the term ``cloud'' corresponds to the Virtual
Organization Cluster (VOC) model \cite{Murphy2009, Murphy2009a,Murphy2010},
which enables the creation of virtual cluster environments that are compatible
across sites, transparent to end users, implementable in both phased and
non-disruptive manners, optionally customizable by Virtual Organizations
(VOs), and designed according an an architectural specification. To establish
a VOC, a Virtual Organization (VO) starts virtual machines over a wide area
network through multiple providers, then a network overlay is created amongst
the VM instances. Since the virtual machines used in the VOC can be spawned
from a single image, VOC environments are nominally homogeneous across grid
sites. The providers used for the virtual machines may include existing grid
sites with virtualization capabilities and corporate cloud providers. It
has been demonstrated that a VOC can be created using resources provided
simultaneously by an Open Science Grid (OSG) site, Amazon EC2 instances, and
CERN resources \cite{Stout10}. 

The VOC provides elasticity in its capabilities
by automatically growing or shrinking the number of provisioned VMs in the
cluster based on the state and length of the job queues, allowing the VOC to
operate transparently to the end user without explicit resource reservation
requests; the transparency also applies to both the VO deploying the VOC and
other entities which choose not to use VOCs, which enables the addition of VOC
implementations to existing production grid sites without disruption of current
operational infrastructure. 

The method by which the VOC is able to autonomically
provision itself to adapt to changing workload demands is through the use of
pilot jobs which obtain the physical resources required from other grid sites
or cloud providers. A pilot job starts a virtual machine, which may then be
dynamically configured, or ``contextualized'' \cite{Keahey2008}, at boot time.
The new VM once operational may then contact a dedicated server provided by
the VO to locate the VOC and join the cluster and its overlay network. The
mechanisms for creating the overlay network for spanning sites over a wide area
network vary by implementation, but include IPOP \cite{Ganguly2006}, Violin
\cite{Ruth2005} and ViNe \cite{Tsugawa2006}.

However, the VOC must still rely either on the ability of the VM resources to
lease scheduling mechanisms from the physical site or on the use of virtual
networks which enable the VO to run its own standard job scheduler. Dedicated
grid interconnection middleware such as Globus \cite{Foster1997} can be used
to permit a VOC to join a grid as a site dedicated to the use of a single
VO; other methods include direct submission mechanisms allowing VO members
to submit workloads to the VOC without the use of grid middleware. While a 
general purpose scheduler such as Condor \cite{Tannenbaum2001} may be used for
this purpose, if the jobs to be executed by the virtual cluster do not require
simultaneous interprocess communications, such as bag-of-tasks applications
\cite{Beaumont2008}, then a scheduling system which can operate directly over
a wide area network, such as the Kestrel system described herein, can be deployed
without the need for a traditional, general purpose overlay network. For the
purpose of this work, the cases where virtual networks are not used or cannot
be used due to policy constraints by grid administrators and where a scheduler
provided by the VO requires access to all VM resources regardless of the network
characteristics are considered. 

\section{Job Scheduling and Distribution in Clouds}
While a significant body of work has been done in operating system
virtualization \cite{Barham2003}, virtual networks \cite{Tsugawa2006,Ganguly2006}
and provisioning of virtual machines \cite{Keahey2008,Sotomayor2009,Nurmi2009},
little has been done on job scheduling in Cloud overlays, often assuming
that existing scientific computing job schedulers could be used in
these environments. Clouds built over wide area networks offer great
technical challenges:
\begin{itemize}
\item Cloud providers and grid sites need to trust each other as well as
the image producers in order to instantiate the same images \cite{B'egin2008}.
\item Virtual machine images need to be transfered to all sites involved
as well as made available on all physical nodes through either staging
of shared file systems \cite{Schmidt2010}.
\item Networking of the virtual machine instances need to enable network
agnostic job scheduling.
\item Users need to be taught a new access mechanism or use existing tools
that have been adapted for Clouds.
\end{itemize}
While existing tools \cite{Tannenbaum2001,Deelman2009,Oinn2004,Wilde2009}
can certainly evolve to be utilized in Clouds, this paper presents
a new job scheduling framework specific to cloud computing thanks
to its network agnostic feature. Named Kestrel \cite{Stout2009,Stout10},
this framework is built using the XMPP protocol, a robust standard
for messaging used in large scale instant messaging deployments, and
recently adopted as the only approved instant messaging protocol for
the Department of Defense \cite{DoDXMPP}. The assumptions and design
principles made to create Kestrel were that traditional push models
used in job scheduling did not meet the requirements of the users
who developed their own job overlays on top of the traditional grid
scheduling framework. Instead users developed their own pull model
akin to the BOINC \cite{Anderson2004} middleware and solely used
the grid scheduling to probe remote providers and request resources.
This probing and request mechanism is very similar to the use of a
cloud IaaS API. Examples of such pull based model also known as pilot
job frameworks are PANDA \cite{Maeno2008}, DIRAC \cite{Tsaregorodtsev2004}
and GlideinWMS \cite{Sfiligoi2008}. In addition, virtual networks
will become extremely difficult to create due to the shortage of IPv4
addresses and the slow adoption of IPv6. This means that increasingly,
users will be faced with virtualized resources started behind multiple
layers of NAT. Push based models together with lack of trust from
the site to the image producer will result in an hostile environment
for the virtualized resources. Kestrel aims to tackle this problem
by looking at solutions from the instant messaging world and the XMPP
protocol. XMPP was designed for hostile networking environments as
well as intermittent, heterogeneous resources. XMPP has also demonstrated
its ability to scale to hundreds of thousands of clients through deployments
such as Google Talk and Facebook Chat, a scale that clouds will soon
reach with the many cores era.

When scaling a Virtual Organization Cluster (VOC) \cite{Murphy2009,Murphy2009a,Murphy2010}
from a single site to multiple sites across a Wide Area Network (WAN),
several challenges arise, including Network Address Translation (NAT)
traversal and increased latency. The VOC workload management system
Kestrel has been designed to address these issues and to allow for
easier scaling of VOCs. Thus, to show Kestrel's ability to scale a
VOC across sites, two criteria must be met. The first is that Kestrel
must be able to acknowledge and manage a large number of worker nodes
across NAT boundaries, and the second is that the latencies involved
in cross-site communication should not excessively impede job dispatch
rates.

In high throughput computing, the focus is on long running tasks which
overshadow the time spent during dispatch \cite{Tannenbaum2001,DouglasThain2006}.
However, for many-task computing where many short lived tasks are
typically scheduled \cite{Raicu2008}, the dispatch time can introduce
significant overhead. For Kestrel to scale across sites, it should
not be affected by latency during job dispatch, given tasks that take
on the order of 30 seconds to a minute in duration to execute.


The primary contributions of this work are:
\begin{itemize}
\item XMPP is presented as a fundamental communication protocol for
building distributed systems for scientific computing.
\item Highlight the design decisions behind the multiple iterations
of Kestrel, based on experimental results and feedback from partner
organizations.
\item Performance of Kestrel, and XMPP-based systems in general,
are demonstrated through micro-benchmarks and response delays for
systems spanning grid sites.
\item Large scale results from real-world scientific computing usage
are presented showing Kestrel's potential as a cloud-based job
scheduling overlay framework.
\end{itemize}

The remainder of this work is organized into several chapters covering related
works (Chapter \ref{chap:Related-Work}) and background material on XMPP (Chapter
\ref{chap:XMPP}). The architecture of Kestrel and the design choices behind
its implementation are discussed in Chapter \ref{chap:Kestrel}. Performance
data and results from real world usage in collaboration with STAR are presented
in Chapter \ref{chap:Results}, with final conclusions drawn in Chapter
\ref{chap:Conclusions}.

The motivating scenario is derived from experience
using virtual machines on Clemson University's 10,000 core Palmetto cluster in
KVM user mode, as per administrative policy. The result is a networking environment
in which each virtual machine is 


