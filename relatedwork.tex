\chapter{Related Work}
\label{chap:Related-Work}
Designed to scavenge unused CPU cycles, the Condor High Throughput
Computing (HTC) system executes computationally expensive operations
over time on machines that would otherwise go unused \cite{Tannenbaum2001,DouglasThain2006}.
The architecture of Condor is based on linking processes together
across the various submission and execution nodes in the cluster so
that job data and output may be transferred. To do so, submit and
execute nodes are directly connected to each other, bypassing the
system's central matchmaker, once a job has been matched with an execute
node. While this arrangement is adequate when all of the compute elements
are in the same subnet behind a NAT boundary, using Condor with machines
on both sides of a NAT boundary introduces complications. Falkon \cite{Raicu2007}has
been designed to achieve high dispatch rate on large scale systems
over-performing Condor in situation where resources are pre-allocated
and the scheduling decision is kept to a minimum. Kestrel has similar
features since a Kestrel pool is created once all agents have started
in the Cloud, however Kestrel has a traditional scheduler based on
Condor matchmaking mechanism.

Several workflow engines \cite{Deelman2004,Oinn2004} and scripting
languages \cite{Wilde2009} have been developed to run jobs efficiently
on computational grids; they have mostly been designed with Globus
\cite{Foster1997} grids in mind where the middleware uses gatekeepers
to access resources. These tools can be adapted to use Clouds and
provide a way to optimize workflow based on Cloud utility pricing
\cite{Deelman2009}but they may not fare well in infrastructure without
inbound network connection to virtualized resources and without shared
file system.

The BOINC \cite{Anderson2004} middleware and existing pilot job frameworks
\cite{Maeno2008,Sfiligoi2008,Tsaregorodtsev2004} correspond to a
pull based model that would work well in Clouds. BOINC relies on volunteers
to contribute cycles and on communities to prepare their applications
as a project. While a large number of projects have been created,
BOINC is not considered a general purpose job scheduler. The pilot
job frameworks are the closest to Kestrel. Kestrel differs in the
use of XMPP which is a standard for messaging and offers great scaling.

A solution to allow Condor to operate with NAT traversal is the Internet
Protocol over Peer-to-Peer system, or IPOP \cite{Ganguly2006}. IPOP
provides a generic networking stack with an underlying P2P network
for data transmission. Unlike the regular networking stack on the
host machine, IPOP does not require any centralized control or routing
configuration. As a result, the pool of IPOP peers forms a virtual
network that is able to span across NAT boundaries \cite{Ganguly2006a}.
Since it is a full networking stack, IPOP provides TCP, UDP, and TLS/SSL
network transports; end-user applications may run unmodified while
using IPOP. It was used to build an autonomic Cloud overlay following
the VOC model in \cite{Abraham2010}. Other virtual networks have
been studied but not used in Cloud overlays \cite{Ruth2005,Tsugawa2006}.

While working on optimizing meander line radio frequency identification
(RFID) antennas, Weis and Lewis \cite{Weis2009} developed a parallelized
computation system based on XMPP. The system was formed from a static
pool of available hardware machines, XMPP clients for each worker
machine, and an XMPP-based scheduler. The scheduler bootstrapped the
system by contacting each machine in the list of workers through SSH
and starting an XMPP agent if one was not already running. The end
result was a specific purpose, ad-hoc grid system. Kestrel uses a
similar architecture, but is intended as a more general purpose job
distribution and scheduling framework. At a similar time XMPP was
used in Bioinformatics web services \cite{Wagener2009} and used within
the Taverna workflow engine \cite{Oinn2004} showing that other communities
are considering XMPP as a suitable protocol for messaging and using
it to develop new services and mechanisms.
