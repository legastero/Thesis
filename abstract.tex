\chapter*{Abstract}

A new distributed computing framework, named Kestrel, for Many-Task Computing
(MTC) applications and implementing Virtual Organization Clusters (VOCs) is
proposed. Kestrel is a lightweight, highly available system based on the
Extensible Messaging and Presence Protocol (XMPP), and has been developed to
explore XMPP-based techniques for improving MTC and VOC tolerance to faults due
to scaling and intermittently connected heterogeneous resoureces. Kestrel provides
a VOC with a special purpose scheduler for VOCs which can provide better scalability
under certain workload assumptions, namely CPU bound processes and bag-of-task
applications.

Experimental results have shown that Kestrel is capable of operating
a VOC of at least 1600 worker nodes with all nodes visible to the
scheduler at once. When using multiple sites located in both North
America and Europe, the latencies introduced to the round trip time
of messages were on the order of 0.3 seconds. To offset the overhead
of XMPP processing, a task execution time of 2 seconds is sufficient
for a pool of 900 workers on a single site to operate at near 100\%
use. Requiring tasks that take on the order of 30 seconds to a minute
to execute would compensate for increased latency during job dispatch
across multiple sites.

Kestrel's architecture is rooted in pilot job frameworks heavily used in Grid
computing, it is also modeled after the use of IRC by botnets to communica te
between compromised machines and command and control servers. For Kestrel, the
extensibility of XMPP has allowed development of protocols for identifying
manager nodes, discovering the capabilities of worker agents, and for
distributing tasks. The presence notifications provided by XMPP allow Kestrel
to monitor the global state of the pool and to perform task dispatching based
on worker availability. In this work it is argued that XMPP is by design a very
good fit for cloud computing frameworks. It offers scalability, federation
between servers and some autonomicity of the agents.

During the summer of 2010, Kestrel was used and modified based on feedback from
the STAR group at Brookhaven National Laboratories. STAR provided a virtual
machine image with applications for simulating proton collisions using PYTHIA
and GEANT3. A Kestrel-based virtual organization cluster, created on top of
Clemson University's Palmetto cluster, was able to provide over 400,000 CPU
hours of computation over the course of a month using an average of 800 virtual
machine instances every day, generating nearly seven terabytes of data and the
largest PYTHIA production run that STAR ever achieved. Several architectural
issues were encountered during the course of the experiment and were resolved
by moving from the original JSON protocols used by Kestrel to native XMPP
equivalents that offered better message delivery confirmation and integration
with existing tools.
